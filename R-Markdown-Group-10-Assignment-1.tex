% !TeX root = RJwrapper.tex
\title{Predicting Motor Vehicle Collisions in New York City}
\author{by Alex Fung, Viswesh Krishnamurthy, Tony Lee, Patrick Osborne}

\maketitle


\hypertarget{abstract}{%
\subsection{Abstract}\label{abstract}}

Technological progress in the world has unarguably improved the quality
of life for the average person in many ways. The age of the automobile
has shaped the way in which work, play and live our lives. Roadways,
buildings, cities and entire countries have been designed to accommodate
motor vehicles. As automobile technology has advanced making cars faster
and capable of more advanced maneuvers, so has our concern with the
safety of these vehicles. Entire disciplines such as traffic management
are devoted to optimizing numerous factors to ensure the safe and
efficient movement of people and goods. As we move into the age of data,
all stakeholders in the automobile industry must effectively collect and
utilize the wealth of information available to better meet their goals
if progress is to continue. In this project, we take the position of a
law enforcement agency, the New York City Police Department, as they
seek to best utilize their resources in the context of responding to
traffic collisions in the city.

\hypertarget{background}{%
\subsection{Background}\label{background}}

At the end of 2017 in New York City, there were 1,923,041 cars
registered to residents of the city.
(\url{https://nyc.streetsblog.org/2018/10/03/car-ownership-continues-to-rise-under-mayor-de-blasio/})
This already-significant number does not include the heavy flow of
vehicles of those who visit the city or are simply passing through. By
contrast, the New York City Police Department (NYPD) budgets for a
headcount of 35,822 uniformed officers
(\url{http://council.nyc.gov/budget/wp-content/uploads/sites/54/2017/03/056-NYPD.pdf}
- page 4), distributed across 77 police precincts (geographic divisions
of the city). On-duty officers/traffic enforcement agents are allocated
to each precinct to enforce traffic laws and handle emergency and
administrative response to traffic incidents (such as collisions). NYC
has been collecting traffic data, including specific data on vehicle
collisions since 2014 to support ``Vision Zero'' , a traffic safety
initiative which has the goal of eliminating traffic fatalities.
(\url{https://data.cityofnewyork.us/Public-Safety/Motor-Vehicle-Collisions-Crashes/h9gi-nx95/data})

\hypertarget{objective}{%
\subsection{Objective}\label{objective}}

The objective of our analysis is to develop a supervised, prediction
model using Machine Learning techniques and the CRISP-DM framework (cite
textbook) on the available collision data to predict whether there will
be a collision in a specified police precinct at a specified time. The
intent of predicting this data is to inform the NYPD's optimal
assignment of limited officers and resources across the 77 police
precincts.

\hypertarget{data-analysis}{%
\subsection{Data Analysis}\label{data-analysis}}

The data set that supports this analysis is sourced from the NYC Open
Data project. The title of the data set is ``Motor Vehicle Collisions --
Crashes''. It contains entries for every collision recorded within New
York City limits by NYPD agents beginning July 1st, 2012 up to the
present day. There are approximately 1.65 million entries in the data
set.

\hypertarget{data-dictionary}{%
\subsection{Data Dictionary}\label{data-dictionary}}

{[}NOTE ON EXCLUDED VARIABLES TO BE INCLUDED{]}

Data dictionary sourced from
\url{https://data.cityofnewyork.us/Public-Safety/Motor-Vehicle-Collisions-Crashes/h9gi-nx95/data}
- ``MVCollisionsDataDictionary\_20190813\_ERD.xlsx''.

\begin{table}

\caption{\label{tab:unnamed-chunk-1}Data Dictionary - Motor Vehicle Collisions – Crashes}
\centering
\begin{tabular}[t]{l|l}
\hline
Feature & Feature.Description\\
\hline
COLLISION\_ID & Unique record code generated by system\\
\hline
ACCIDENT\_DATE & Occurrence date of collision\\
\hline
ACCIDENT\_TIME & Occurrence time of collision\\
\hline
BOROUGH & Borough where collision occurred\\
\hline
ZIP CODE & Postal code of incident occurrence\\
\hline
LATITUDE & Latitude coordinate for Global Coordinate System, WGS 1984, decimal degrees (EPSG 4326)\\
\hline
LONGITUDE & Longitude coordinate for Global Coordinate System, WGS 1984, decimal degrees (EPSG 4326)\\
\hline
LOCATION & Latitude , Longitude pair\\
\hline
ON STREET NAME & Street on which the collision occurred\\
\hline
CROSS STREET NAME & Nearest cross street to the collision\\
\hline
OFF STREET NAME & Street address if known\\
\hline
NUMBER OF PERSONS INJURED & Number of persons injured\\
\hline
NUMBER OF PERSONS KILLED & Number of persons killed\\
\hline
NUMBER OF PEDESTRIANS INJURED & Number of pedestrians injured\\
\hline
NUMBER OF PEDESTRIANS KILLED & Number of pedestrians killed\\
\hline
NUMBER OF CYCLIST INJURED & Number of cyclists injured\\
\hline
NUMBER OF CYCLIST KILLED & Number of cyclists killed\\
\hline
NUMBER OF MOTORIST INJURED & Number of vehicle occupants injured\\
\hline
NUMBER OF MOTORIST KILLED & Number of vehicle occupants killed\\
\hline
CONTRIBUTING FACTOR VEHICLE 1 & Factors contributing to the collision for designated vehicle\\
\hline
CONTRIBUTING FACTOR VEHICLE 2 & Factors contributing to the collision for designated vehicle\\
\hline
CONTRIBUTING FACTOR VEHICLE 3 & Factors contributing to the collision for designated vehicle\\
\hline
CONTRIBUTING FACTOR VEHICLE 4 & Factors contributing to the collision for designated vehicle\\
\hline
CONTRIBUTING FACTOR VEHICLE 5 & Factors contributing to the collision for designated vehicle\\
\hline
VEHICLE TYPE CODE 1 & Type of vehicle based on the selected vehicle category (ATV, bicycle, car/suv, ebike, escooter, truck/bus, motorcycle, other)\\
\hline
VEHICLE TYPE CODE 2 & Type of vehicle based on the selected vehicle category (ATV, bicycle, car/suv, ebike, escooter, truck/bus, motorcycle, other)\\
\hline
VEHICLE TYPE CODE 3 & Type of vehicle based on the selected vehicle category (ATV, bicycle, car/suv, ebike, escooter, truck/bus, motorcycle, other)\\
\hline
VEHICLE TYPE CODE 4 & Type of vehicle based on the selected vehicle category (ATV, bicycle, car/suv, ebike, escooter, truck/bus, motorcycle, other)\\
\hline
VEHICLE TYPE CODE 5 & Type of vehicle based on the selected vehicle category (ATV, bicycle, car/suv, ebike, escooter, truck/bus, motorcycle, other)\\
\hline
\end{tabular}
\end{table}

\hypertarget{default-r-markdown-code-below}{%
\section{DEFAULT R MARKDOWN CODE
BELOW}\label{default-r-markdown-code-below}}

\hypertarget{r-markdown}{%
\subsection{R Markdown}\label{r-markdown}}

This is an R Markdown document. Markdown is a simple formatting syntax
for authoring HTML, PDF, and MS Word documents. For more details on
using R Markdown see \url{http://rmarkdown.rstudio.com}.

When you click the \textbf{Knit} button a document will be generated
that includes both content as well as the output of any embedded R code
chunks within the document. You can embed an R code chunk like this:

\begin{Schunk}
\begin{Sinput}
summary(cars)
\end{Sinput}
\begin{Soutput}
#>      speed           dist       
#>  Min.   : 4.0   Min.   :  2.00  
#>  1st Qu.:12.0   1st Qu.: 26.00  
#>  Median :15.0   Median : 36.00  
#>  Mean   :15.4   Mean   : 42.98  
#>  3rd Qu.:19.0   3rd Qu.: 56.00  
#>  Max.   :25.0   Max.   :120.00
\end{Soutput}
\end{Schunk}

\hypertarget{including-plots}{%
\subsection{Including Plots}\label{including-plots}}

You can also embed plots, for example:

\begin{Schunk}

\includegraphics{R-Markdown-Group-10-Assignment-1_files/figure-latex/pressure-1} \end{Schunk}

Note that the \texttt{echo\ =\ FALSE} parameter was added to the code
chunk to prevent printing of the R code that generated the plot.

\hypertarget{document-style-attribution}{%
\subsection{Document Style
Attribution}\label{document-style-attribution}}

This document was generated using a modified version of the
``RJournal.sty'' file provided by the The R Foundation at
\url{https://journal.r-project.org/submissions.html}.

The document can be regenerated in RStudio by Knitting the provided ``R
Markdown-Group 10-Assignment 1.Rmd'' file with the provided
``RJournal.sty'' file in the same directory.


\address{%
Alex Fung\\
\\
\\
}


\address{%
Viswesh Krishnamurthy\\
\\
\\
}


\address{%
Tony Lee\\
\\
\\
}


\address{%
Patrick Osborne\\
\\
\\
}


